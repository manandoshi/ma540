\documentclass{article}
\usepackage[utf8]{inputenc}
\usepackage[margin=1in]{geometry}
\title{Assignment 2: $\Theta$-method for Heat equation}
\author{Manan Doshi}
\date{11 March 2018}
\usepackage{graphicx}
\usepackage{amsmath}

\begin{document}
\graphicspath{{/}}
\maketitle

\section{Introduction}
The $\theta$ method is a semi-implicit forward time centered space scheme to solve the heat equation. The heat equation in one dimension is given as follows:
\[
\frac{\partial u}{\partial t} = \frac{\partial^2 u}{\partial x^2}
\]

\section{$\Theta$ Scheme}
The scheme to solve the equation is as follows:

\begin{align*}
    \partial_t^{+}U_j^k &= (\theta) \partial_x^{+}\partial_x^{-}U_j^{k+1} + (1-\theta) \partial_x^{+}\partial_x^{-}U_j^k \\[1em]
    \left(\frac{U_j^{k+1} - U_j^k}{\Delta t}\right) &= (\theta) \left(\frac{U^{k+1}_{j+1} - 2 U_j^{k+1} + U^{k+1}_{j-1}}{\Delta x^2}\right) 
                    \\&+ (1-\theta)\left(\frac{U^{k}_{j+1} - 2 U_j^{k} + U^{k}_{j-1}}{\Delta x^2}\right) \\
\end{align*}

Setting $\lambda_1 = \left(\frac{\Delta t}{\Delta x^2}\right) \theta$ and $\lambda_2 = \left(\frac{\Delta t}{\Delta x^2}\right) (\theta-1)$, we can write it in a matrix-vector form,
\[
    \begin{bmatrix}
        1 + 2\lambda_1 & -\lambda_1 & 0 &  \ldots & 0\\
        -\lambda_1 & 1 + 2\lambda_1 & -\lambda_1 &  \ddots & 0 \\
        0 & -\lambda_1 & 1 + 2\lambda_1 &  \ddots & 0 \\
        \vdots & \ddots & \ddots & \ddots & \vdots \\
        0 & \ldots & 0 & -\lambda_1 & 1 + 2\lambda_1  \\
    \end{bmatrix}
    \begin{bmatrix}
        U_1^{k+1}\\
        U_2^{k+1}\\
        \vdots\\
        \vdots\\
        U_{J-1}^{k+1}
    \end{bmatrix}
    =
    \begin{bmatrix}
        1 + 2\lambda_2 & -\lambda_2 & 0 &  \ldots & 0\\
        -\lambda_2 & 1 + 2\lambda_2 & -\lambda_2 &  \ddots & 0 \\
        0 & -\lambda_2 & 1 + 2\lambda_2 &  \ddots & 0 \\
        \vdots & \ddots & \ddots & \ddots & \vdots \\
        0 & \ldots & 0 & -\lambda_2 & 1 + 2\lambda_2  \\
    \end{bmatrix}
    \begin{bmatrix}
        U_1^{k}\\
        U_2^{k}\\
        \vdots\\
        \vdots\\
        U_{J-1}^{k}
    \end{bmatrix}
\]

Taking the inverse, we can write it as a linear equation
\[
    U^{k+1} = {\bf A(\mu,\theta)}U^k
\]

The eigenvalues of the matrix ${\bf A}$ dictate the stability of the scheme.

\newpage
\section{Stability Analysis using eigenvalues}

It can be clearly seen in \figurename{\ref{eig}} that the method is stable for $\theta \geq 0.5$ and for $\theta < 0.5$ when $\mu < 0.5$. This is what we obtained from Von-Neumann Stability analysis of the method.


\begin{figure}[h!]
    \centering
    \includegraphics[scale=0.6]{eig}
    \caption{Stability plot fot $\theta$ method}
    \label{eig}
\end{figure}

\section{Convergence study}
The following set of plots show the solution for various values of $\mu$ and $\theta$. The red line represents the exact solution and the blue line represents the numerical solution. The convergence plots are log log plotted against $\mu$ and $\Delta t$. The red line in this plot represents $\mu=0.5$. The contour plot represents the solution in the $x$-$t$ plane.

\newpage
\subsection{Explicit ($\theta = 0$)}
\begin{figure}[h!]
    \centering
    \includegraphics[scale=0.7]{theta0}
    \caption{Convergence plot for $\theta=0$}
\end{figure}

\begin{figure}[h!]
    \centering
    \includegraphics[scale=0.7]{mu2theta0}
    \caption{Solution for $\mu=0.2$ and $\theta=0$}
\end{figure}
\begin{figure}[h!]
    \centering
    \includegraphics[scale=0.7]{c_mu2theta0}
    \caption{Solution for $\mu=0.2$ and $\theta=0$}
\end{figure}
\begin{figure}[h!]
    \centering
    \includegraphics[scale=0.7]{mu6theta0}
    \caption{Solution for $\mu=0.6$ and $\theta=0$ (Blowup. Unstable)}
\end{figure}

\clearpage
\subsection{Implicit ($\theta = 1$)}
\begin{figure}[h!]
    \centering
    \includegraphics[scale=0.7]{theta10}
    \caption{Convergence plot for $\theta=1$}
\end{figure}

\begin{figure}[h!]
    \centering
    \includegraphics[scale=0.7]{mu2theta10}
    \caption{Solution for $\mu=0.2$ and $\theta=1$}
\end{figure}
\begin{figure}[h!]
    \centering
    \includegraphics[scale=0.7]{c_mu2theta10}
    \caption{Solution for $\mu=0.2$ and $\theta=1$}
\end{figure}
\begin{figure}[h!]
    \centering
    \includegraphics[scale=0.7]{mu20theta10}
    \caption{Solution for $\mu=2.0$ and $\theta=1$}
\end{figure}

\clearpage
\subsection{CN ($\theta = 0.5$)}
\begin{figure}[h!]
    \centering
    \includegraphics[scale=0.7]{theta5}
    \caption{Convergence plot for $\theta=0.5$}
\end{figure}

\begin{figure}[h!]
    \centering
    \includegraphics[scale=0.7]{mu2theta5}
    \caption{Solution for $\mu=0.2$ and $\theta=0.5$}
\end{figure}
\begin{figure}[h!]
    \centering
    \includegraphics[scale=0.7]{c_mu2theta5}
    \caption{Solution for $\mu=0.2$ and $\theta=0.5$}
\end{figure}
\begin{figure}[h!]
    \centering
    \includegraphics[scale=0.7]{mu6theta5}
    \caption{Solution for $\mu=0.6$ and $\theta=0.5$}
\end{figure}

\clearpage
\subsection{$\theta = 0.6$}
\begin{figure}[h!]
    \centering
    \includegraphics[scale=0.7]{theta6}
    \caption{Convergence plot for $\theta=0.6$}
\end{figure}

\begin{figure}[h!]
    \centering
    \includegraphics[scale=0.7]{mu2theta6}
    \caption{Solution for $\mu=0.2$ and $\theta=0.6$}
\end{figure}
\begin{figure}[h!]
    \centering
    \includegraphics[scale=0.7]{c_mu2theta6}
    \caption{Solution for $\mu=0.2$ and $\theta=0.6$}
\end{figure}
\begin{figure}[h!]
    \centering
    \includegraphics[scale=0.7]{mu20theta6}
    \caption{Solution for $\mu=2.0$ and $\theta=0.6$}
\end{figure}

\clearpage
\subsection{$\theta = 0.4$}
\begin{figure}[h!]
    \centering
    \includegraphics[scale=0.7]{theta4}
    \caption{Convergence plot for $\theta=0.4$}
\end{figure}

\begin{figure}[h!]
    \centering
    \includegraphics[scale=0.7]{mu2theta4}
    \caption{Solution for $\mu=0.2$ and $\theta=0.4$}
\end{figure}
\begin{figure}[h!]
    \centering
    \includegraphics[scale=0.7]{c_mu2theta4}
    \caption{Solution for $\mu=0.2$ and $\theta=0.4$}
\end{figure}
\begin{figure}[h!]
    \centering
    \includegraphics[scale=0.7]{mu20theta4}
    \caption{Solution for $\mu=2.0$ and $\theta=0.4$ (Unstable)}
\end{figure}

\section{Conclusion}
We have successfully checked the convergence and stability of the $\theta$ method for the heat equation for various values of $\mu$ and $\theta$. We have shown that the sceme is stable for large time steps when $\theta \geq 0.5$ and is conditionally stable otherwise.
\end{document}
